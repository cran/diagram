\documentclass[article,nojss]{jss}
\DeclareGraphicsExtensions{.pdf,.eps}

%%%%%%%%%%%%%%%%%%%%%%%%%%%%%% Add-on packages and fonts
\usepackage{amsmath}
\usepackage{xspace}
\usepackage{verbatim}
\usepackage[english]{babel}
%\usepackage{mathptmx}
%\usepackage{helvet}
\usepackage[T1]{fontenc}
\usepackage[latin1]{inputenc}

%%%%%%%%%%%%%%%%%%%%%%%%%%%%%% User specified LaTeX commands.
\newcommand{\di}{\textbf{\textsf{diagram}}\xspace}

\title{\proglang{R} Package \pkg{diagram}: visualising simple graphs,
flowcharts, and webs}

\Plaintitle{R Package diagram: visualising simple graphs, flowcharts, and webs}

\Keywords{diagram, food web, flow chart, arrows, \proglang{R}}

\Plainkeywords{diagram, food web, flow chart, arrows, R}


\author{Karline Soetaert\\
Centre for Estuarine and Marine Ecology\\
Netherlands Institute of Ecology\\
The Netherlands
}

\Plainauthor{Karline Soetaert}

\Abstract{This document describes how to use the \pkg{diagram} package
for plotting small networks, flow charts, and (food) webs.

Together with R-package \pkg{shape} \citep{Soetaert08f} this package has
been written to produce the figures of the book \citep{Soetaert08}
 }

%% The address of (at least) one author should be given
%% in the following format:
\Address{
  Karline Soetaert\\
  Centre for Estuarine and Marine Ecology (CEME)\\
  Netherlands Institute of Ecology (NIOO)\\
  4401 NT Yerseke, Netherlands
  E-mail: \email{k.soetaert@nioo.knaw.nl}\\
  URL: \url{http://www.nioo.knaw.nl/ppages/ksoetaert}\\
}


%%%%%%%%%%%%%%%%%%%%%%%%%%%%%% R/Sweave specific LaTeX commands.
%% need no \usepackage{Sweave}
%\VignetteIndexEntry{diagram}

%%%%%%%%%%%%%%%%%%%%%%%%%%%%%% Begin of the document
\begin{document}



\maketitle

\section{Introduction}

There are three ways in which package \pkg{diagram} can be used:

\begin{itemize}
\item function \code{plotmat} takes as input a matrix with transition coefficients
or interaction strengths. It plots the corresponding network consisting of
(labeled) boxes (the components) connected by arrows. Each arrow is labeled with the value
of the coefficients.

\item function \code{plotweb} takes as input a matrix with (flow) values, and plots a
web. Here the components are connected by arrows whose thickness is determined by the
value of the coefficients.

\item Flowcharts can be made by adding separate objects (textboxes) to the figure and connecting these with arrows.
\end{itemize}

Three datasets have been included:
\begin{itemize}
\item \code{Rigaweb}, the planktonic food web of the Gulf of Riga \citep{Donali99}.
\item \code{Takapotoweb}, the Takapoto atoll planktonic food web \citep{Niquil98}.
\item \code{Teasel}, the transition matrix describing the population dynamics of Teasel, a European perennial weed (\citep{Caswell01}, \citep{Soetaert08}.
\end{itemize}

The food webs were generated using \proglang{R} packages \pkg{LIM} and
\pkg{limSolve} \citep{Soetaert08b, Soetaert08c} which contain functions to read and
solve food web problems respectively.

\section{plotmat - plotting networks based on matrix input}

This is the quickest method of plotting a network. The network is specified in
a matrix, which gives the magnitudes of the links (from columns to rows).

The position of the elements (boxes) is specified by argument \code{pos}.
Thus, setting \code{pos=c(1,2,1)} indicates that the 4 elements will be arranged
in three equidistant rows; on the first row one element, on the second row two elements
and on the third row one element.

\subsection{Simple examples}
\begin{Schunk}
\begin{Sinput}
> par(mar=c(1,1,1,1),mfrow=c(2,2))
> #
> # first graph - four simple boxes
> #
> names <- c("A","B","C","D")
> M <- matrix(nrow=4,ncol=4,byrow=TRUE,data=0)
> pp<-plotmat(M,pos=c(1,2,1),name=names,lwd=1,box.lwd=2,cex.txt=0.8,
+             box.size=0.1,box.type="square",box.prop=0.5)
> #
> # second graph - round boxes with arrows, labeled "flow"
> #
> M[2,1]<-M[3,1]<-M[4,2]<-M[4,3] <- "flow"
> pp<-plotmat(M,pos=c(1,2,1),curve=0,name=names,lwd=1,box.lwd=2,cex.txt=0.8,
+             box.type="circle",box.prop=1.0)
> #
> # third graph - diamond-shaped boxes including self-arrows
> #
> diag(M) <- "self"
> pp<-plotmat(M,pos=c(2,2),curve=0,name=names,lwd=1,box.lwd=2,cex.txt=0.8,
+             self.cex=0.5,self.shiftx=c(-0.1,0.1,-0.1,0.1),
+             box.type="diamond",box.prop=0.5)
> #
> # fourth graph - hexagonal-shaped boxes, curved arrows
> #
> M <- matrix(nrow=4,ncol=4,data=0)
> M[2,1]<-1  ;M[4,2]<-2;M[3,4]<-3;M[1,3]<-4
> pp<-plotmat(M,pos=c(1,2,1),curve=0.2,name=names,lwd=1,box.lwd=2,cex.txt=0.8,
+             arr.type="triangle",box.size=0.1,box.type="hexa",box.prop=0.5)
> mtext(outer=TRUE,side=3,line=-1.5,cex=1.5,"plotmat")
> #
> par(mfrow=c(1,1))
\end{Sinput}
\end{Schunk}
\setkeys{Gin}{width=0.6\textwidth}
\begin{figure}
\begin{center}
\includegraphics{diagram-fig1}
\end{center}
\caption{Four simple examples of \code{plotmat}}
\label{fig:one}
\end{figure}

\subsection{a schematic representation of an ecosystem model}
In the example below, first the main components and arrows are drawn
(\code{plotmat}), and the output of this function written in list pp.
This contains, a.o. the positions of the components (boxes), arrows, etc..
It is used to draw an arrow from the middle of the arrow connecting fish and zooplankton
("ZOO") to detritus. Function \code{straightarrow} (see below) is used to draw this arrow.
\begin{Schunk}
\begin{Sinput}
> names <- c("PHYTO","NH3","ZOO","DETRITUS","BotDET","FISH")
> M <- matrix(nrow=6,ncol=6,byrow=TRUE,data=c(
+ #   p n z  d  b  f
+     0,1,0, 0, 0, 0, #p
+     0,0,4, 10,11,0, #n
+     2,0,0, 0, 0, 0, #z
+     8,0,13,0, 0, 12,#d
+     9,0,0, 7, 0, 0, #b
+     0,0,5, 0, 0, 0  #f
+     ))
> #
> pp<-plotmat(M,pos=c(1,2,1,2),curve=0,name=names,lwd=1,box.lwd=2,cex.txt=0.8,
+             box.type="square",box.prop=0.5,arr.type="triangle",
+             arr.pos=0.4,shadow.size=0.01,prefix="f",
+             main="NPZZDD model")
> #
> phyto   <-pp$comp[names=="PHYTO"]
> zoo     <-pp$comp[names=="ZOO"]
> nh3     <-pp$comp[names=="NH3"]
> detritus<-pp$comp[names=="DETRITUS"]
> fish    <-pp$comp[names=="FISH"]
> #
> # flow5->detritus
> #
> m2 <- 0.5*(zoo+fish)
> m1 <- detritus
> m1[1] <- m1[1]+ pp$radii[4,1]
> mid <- straightarrow (to=m1,from=m2,arr.type="triangle",arr.pos=0.4,lwd=1)
> text(mid[1],mid[2]+0.03,"f6",cex=0.8)
> #
> # flow2->detritus
> #
> m2 <- 0.5*(zoo+phyto)
> m1 <- detritus
> m1[1] <-m1[1] + pp$radii[3,1]*0.2
> m1[2] <-m1[2] + pp$radii[3,2]
> mid<-straightarrow (to=m1,from=m2,arr.type="triangle",arr.pos=0.3,lwd=1)
> text(mid[1]-0.01,mid[2]+0.03,"f3",cex=0.8)
\end{Sinput}
\end{Schunk}
\setkeys{Gin}{width=0.6\textwidth}
\begin{figure}
\begin{center}
\includegraphics{diagram-fig2}
\end{center}
\caption{An NPZZDD model}
\label{fig:two}
\end{figure}

\subsection{Plotting a transition matrix}
The next example uses formulae to label the arrows \footnote{This is now possible thanks to Yvonnick Noel, Univ. Rennes, France}. This is done by passing
a data.frame rather than a matrix to function \code{plotmat}
\begin{Schunk}
\begin{Sinput}
> # Create population matrix
> #
> Numgenerations   <- 6
> DiffMat  <- matrix(data=0,nrow=Numgenerations,ncol=Numgenerations)
> AA <- as.data.frame(DiffMat)
> AA[[1,4]]<- "f[3]"
> AA[[1,5]]<- "f[4]"
> AA[[1,6]]<- "f[5]"
> #
> AA[[2,1]]<- "s[list(0,1)]"
> AA[[3,2]]<- "s[list(1,2)]"
> AA[[4,3]]<- "s[list(2,3)]"
> AA[[5,4]]<- "s[list(3,4)]"
> AA[[6,5]]<- "s[list(4,5)]"
> #
> name  <- c(expression(Age[0]),expression(Age[1]),expression(Age[2]),
+            expression(Age[3]),expression(Age[4]),expression(Age[5]))
> #
> PP <- plotmat(A=AA,pos=6,curve=0.7,name=name,lwd=2,arr.len=0.6,arr.width=0.25,my=-0.2,
+               box.size=0.05,arr.type="triangle",dtext= 0.95,
+               main="Age-structured population model 1")
\end{Sinput}
\end{Schunk}
\setkeys{Gin}{width=0.6\textwidth}
\begin{figure}
\begin{center}
\includegraphics{diagram-fig3}
\end{center}
\caption{A transition matrix}
\label{fig:three}
\end{figure}

\subsection{Another transition matrix}
The data set \code{Teasel} contains the transition matrix of the population dynamics model of teasel
(Dipsacus sylvestris), a European perennial weed, \citep{Caswell01}, \citep{Soetaert08}
\begin{Schunk}
\begin{Sinput}
> Teasel
\end{Sinput}
\begin{Soutput}
         DS 1yr DS 2yr R small R medium R large       F
DS 1yr    0.000   0.00   0.000    0.000   0.000 322.380
DS 2yr    0.966   0.00   0.000    0.000   0.000   0.000
R small   0.013   0.01   0.125    0.000   0.000   3.448
R medium  0.007   0.00   0.125    0.238   0.000  30.170
R large   0.008   0.00   0.038    0.245   0.167   0.862
F         0.000   0.00   0.000    0.023   0.750   0.000
\end{Soutput}
\end{Schunk}
This dataset is plotted using curved arrows; we specify the curvature in a matrix
called \code{curves}.
\begin{Schunk}
\begin{Sinput}
> curves <- matrix(nrow=ncol(Teasel),ncol=ncol(Teasel),0)
> curves[3,1]<- curves[1,6]<- -0.35
> curves[4,6]<- curves[6,4]<- curves[5,6]<- curves[6,5]<-0.08
> curves[3,6]<-  0.35
> plotmat(Teasel,pos=c(3,2,1),curve=curves,name=colnames(Teasel),lwd=1,box.lwd=2,cex.txt=0.8,
+         box.cex=0.8,box.size=0.08,arr.length=0.5,box.type="circle",box.prop=1,
+         shadow.size = 0.01,self.cex=0.6,my=-0.075, mx=-0.01,relsize=0.9,
+         self.shiftx=c(0,0,0.125,-0.12,0.125,0),self.shifty=0,main="Teasel population model")
\end{Sinput}
\end{Schunk}
\setkeys{Gin}{width=1.0\textwidth}
\begin{figure}
\begin{center}
\includegraphics{diagram-fig4}
\end{center}
\caption{The Teasel data set}
\label{fig:four}
\end{figure}

\section{plotweb - plotting webs based on matrix input}
Given a matrix containing flow values (from rows to columns), function \code{plotweb}
will plot a web. The elements are positioned on a circle, and connected by arrows;
the magnitude of web flows determines the thickness of the arrow.

This function is less flexible than \code{plotmat}.
\begin{Schunk}
\begin{Sinput}
> BB <- matrix(nrow=20,ncol=20,1:20)
> diag(BB)<-0
> plotweb(BB,legend=TRUE,maxarrow=3)
> par(mfrow=c(1,1))
\end{Sinput}
\end{Schunk}
\setkeys{Gin}{width=0.8\textwidth}
\begin{figure}
\begin{center}
\includegraphics{diagram-fig5}
\end{center}
\caption{Plotweb}
\label{fig:five}
\end{figure}

\subsection{Foodwebs}
Dataset \code{Rigaweb} (\citep{Donali99} contains flow values for the food web of the Gulf of Riga planktonic
system.
\begin{Schunk}
\begin{Sinput}
> Rigaweb
\end{Sinput}
\begin{Soutput}
                    P1      P2        B       N        Z        D      DOC       CO2
P1             0.00000  0.0000   0.0000 4.12297 10.49431 0.000000 1.565910  17.22501
P2             0.00000  0.0000   0.0000 0.00000 16.79755 4.457164 2.723090  29.95399
B              0.00000  0.0000   0.0000 9.44000  0.00000 0.000000 0.000000 244.99223
N              0.00000  0.0000   0.0000 0.00000  0.00000 0.000000 0.000000  13.40297
Z              0.00000  0.0000   0.0000 0.00000  0.00000 3.183226 3.963226  30.19580
D              0.00000  0.0000   0.0000 0.00000 12.34039 0.000000 0.000000   0.00000
DOC            0.00000  0.0000 261.1822 0.00000  0.00000 0.000000 0.000000   0.00000
CO2           31.31820 54.4618   0.0000 0.00000  0.00000 0.000000 0.000000   0.00000
Sedimentation  0.00000  0.0000   0.0000 0.00000  0.00000 0.000000 0.000000   0.00000
              Sedimentation
P1                     0.10
P2                     0.34
B                      0.00
N                      0.00
Z                      0.78
D                     13.92
DOC                    0.00
CO2                    0.00
Sedimentation          0.00
\end{Soutput}
\end{Schunk}

\begin{Schunk}
\begin{Sinput}
> plotweb(Rigaweb,main="Gulf of Riga food web",sub="mgC/m3/d",val=TRUE)
\end{Sinput}
\end{Schunk}
\setkeys{Gin}{width=0.8\textwidth}
\begin{figure}
\begin{center}
\includegraphics{diagram-fig6}
\end{center}
\caption{The Gulf of Riga data set}
\label{fig:six}
\end{figure}

\section{functions to create flow charts}

The various functions are given in table (1).
\begin{table*}[t]
\caption{Summary of flowchart functions}\label{tb:arr}
\centering
\begin{tabular}{p{.15\textwidth}p{.75\textwidth}}\\
 Function & Description\\
\hline
openplotmat   &  creates an empty plot\\
coordinates   &  calculates coordinates of elements, neatly arranged in rows/columns\\
bentarrow     &  adds 2-segmented arrow between two points\\
curvedarrow   &  adds curved arrow between two points\\
segmentarrow  &  adds 3-segmented arrow between two points\\
selfarrow     &  adds a circular self-pointing arrow \\
splitarrow    &  adds a branched arrow between several points\\
straightarrow &  adds straight arrow between two points\\
treearrow     &  adds dendrogram-like branched arrow between several points\\
shadowbox     &  adds a box with a shadow to a plot\\
textdiamond   &  adds lines of text in a diamond-shaped box to a plot\\
textellipse   &  adds lines of text in a ellipse-shaped box to a plot\\
textempty     &  adds lines of text on a colored background to a plot\\
texthexa      &  adds lines of text in a hexagonal box to a plot\\
textmulti     &  adds lines of text in a multigonal box to a plot\\
textplain     &  adds lines of text to a plot\\
textrect      &  adds lines of text in a rectangular-shaped box to a plot\\
textround     &  adds lines of text in a rounded box to a plot\\
\hline
\end{tabular}
\end{table*}

The code below generates a flow chart
\begin{Schunk}
\begin{Sinput}
> par(mar=c(1,1,1,1))
> openplotmat()
> elpos  <-coordinates (c(1,1,2,4))
> fromto <- matrix(ncol=2,byrow=TRUE,data=c(1,2,2,3,2,4,4,7,4,8))
> nr     <-nrow(fromto)
> arrpos <- matrix(ncol=2,nrow=nr)
> for (i in 1:nr)
+     arrpos[i,]<- straightarrow (to=elpos[fromto[i,2],],from=elpos[fromto[i,1],]
+         ,lwd=2,arr.pos=0.6,arr.length=0.5)
> textellipse(elpos[1,],0.1,      lab="start",           box.col="green",
+             shadow.col="darkgreen",shadow.size=0.005,cex=1.5)
> textrect   (elpos[2,],0.15,0.05,lab="found term?",     box.col="blue",
+             shadow.col="darkblue",shadow.size=0.005,cex=1.5)
> textrect   (elpos[4,],0.15,0.05,lab="related?",        box.col="blue",
+             shadow.col="darkblue",shadow.size=0.005,cex=1.5)
> textellipse(elpos[3,],0.1,0.1,  lab=c("other","term"), box.col="orange",
+             shadow.col="red",shadow.size=0.005,cex=1.5)
> textellipse(elpos[3,],0.1,0.1,  lab=c("other","term"), box.col="orange",
+             shadow.col="red",shadow.size=0.005,cex=1.5)
> textellipse(elpos[7,],0.1,0.1,  lab=c("make","a link"),box.col="orange",
+             shadow.col="red",shadow.size=0.005,cex=1.5)
> textellipse(elpos[8,],0.1,0.1,  lab=c("new","article"),box.col="orange",
+             shadow.col="red",shadow.size=0.005,cex=1.5)
> #
> dd <- c(0.0,0.025)
> text(arrpos[2,1]+0.05,arrpos[2,2],"yes")
> text(arrpos[3,1]-0.05,arrpos[3,2],"no")
> text(arrpos[4,1]+0.05,arrpos[4,2]+0.05,"yes")
> text(arrpos[5,1]-0.05,arrpos[5,2]+0.05,"no")
\end{Sinput}
\end{Schunk}
\setkeys{Gin}{width=0.8\textwidth}
\begin{figure}
\begin{center}
\includegraphics{diagram-fig7}
\end{center}
\caption{A flow chart}
\label{fig:seven}
\end{figure}

The different types of text boxes are generated with the following code:
\begin{Schunk}
\begin{Sinput}
> openplotmat(main="textbox shapes")
> rx <- 0.1
> ry <- 0.05
> pos <- coordinates(c(1,1,1,1,1,1,1),mx=-0.2)
> textdiamond(mid=pos[1,],radx=rx,rady=ry,lab=LETTERS[1],cex=2,shadow.col="lightblue")
> textellipse(mid=pos[2,],radx=rx,rady=ry,lab=LETTERS[2],cex=2,shadow.col="blue")
> texthexa(mid=pos[3,],radx=rx,rady=ry,lab=LETTERS[3],cex=2,shadow.col="darkblue")
> textmulti(mid=pos[4,],nr=7,radx=rx,rady=ry,lab=LETTERS[4],cex=2,shadow.col="red")
> textrect(mid=pos[5,],radx=rx,rady=ry,lab=LETTERS[5],cex=2,shadow.col="darkred")
> textround(mid=pos[6,],radx=rx,rady=ry,lab=LETTERS[6],cex=2,shadow.col="black")
> textempty(mid=pos[7,],lab=LETTERS[7],cex=2,box.col="yellow")
> pos[,1] <- pos[,1] + 0.5
> text(pos[,1],pos[,2],c("textdiamond","textellipse","texthexa",
+                        "textmulti","textrect","textround","textempty"))
\end{Sinput}
\end{Schunk}
\setkeys{Gin}{width=0.8\textwidth}
\begin{figure}
\begin{center}
\includegraphics{diagram-fig8}
\end{center}
\caption{The text boxes}
\label{fig:eight}
\end{figure}

The different types of arrows are generated with the following code:
\begin{Schunk}
\begin{Sinput}
> par(mar=c(1,1,1,1))
> openplotmat(main="Arrowtypes")
> elpos<-coordinates (c(1,2,1),mx=0.1,my=-0.1)
> curvedarrow(from=elpos[1,],to=elpos[2,],curve=-0.5,lty=2,lcol=2)
> straightarrow(from=elpos[1,],to=elpos[2,],lty=3,lcol=3)
> segmentarrow(from=elpos[1,],to=elpos[2,],lty=1,lcol=1)
> treearrow(from=elpos[2:3,],to=elpos[4,],lty=4,lcol=4)
> bentarrow(from=elpos[3,],to=elpos[3,]-c(0.1,0.1),arr.pos=1,lty=5,lcol=5)
> bentarrow(from=elpos[1,],to=elpos[3,],lty=5,lcol=5)
> selfarrow(pos=elpos[3,],path="R",lty=6,curve=0.075,lcol=6)
> splitarrow(from=elpos[1,],to=elpos[2:3,],lty=1,lwd=1,dd=0.7,arr.side=1:2,lcol=7)
> for ( i in 1:4) textrect (elpos[i,],0.05,0.05,lab=i,cex=1.5)
> legend("topright",lty=1:7,legend=c("segmentarrow","curvedarrow","straightarrow",
+ "treearrow","bentarrow","selfarrow","splitarrow"),lwd=c(rep(2,6),1),col=1:7)
\end{Sinput}
\end{Schunk}
\setkeys{Gin}{width=0.8\textwidth}
\begin{figure}
\begin{center}
\includegraphics{diagram-fig9}
\end{center}
\caption{The arrow types}
\label{fig:nine}
\end{figure}

This vignette was created using Sweave \citep{Leisch02}.

The package is on CRAN, the R-archive website (\citep{R2008})

More examples can be found in the demo's of package \pkg{ecolMod} \citep{Soetaert08e}

\bibliography{vignettes}

\end{document}

